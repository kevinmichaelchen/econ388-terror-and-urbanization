\documentclass[preprint,2p,12pt]{elsarticle}

\usepackage{setspace}
\journal{Professor Sheppard - ECON 388}

\begin{document}

\begin{frontmatter}
\title{Urbanization in Climates of Terror}
\author[kevin]{K.~Chen}
\address[kevin]{Williams College}

\begin{abstract}
The impact of terrorism on cities is a fairly unexplored question.
Past economists have studied the effect of terrorism on urban form.
Specifically, Shepphard finds that an increase in terrorism reduces urban land use, especially in larger cities.
This paper explores how a ``climate'' of terror -- the  persistent threat of terrorism -- affects urbanization rates.
\end{abstract}

\begin{keyword}
terrorism \sep urbanization \sep urban economics
\end{keyword}

\end{frontmatter}

%\onehalfspacing
\doublespacing

\section*{\sc Acknowledgements}
Professor Stephen Sheppard. ``National Consortium for the Study of Terrorism and Responses to Terrorism (START). (2012). Global Terrorism Database [Data file]. Retrieved from http://www.start.umd.edu/gtd''

\section{\sc Introduction}
From an economic perspective, the existence of cities is a result of market equilibria, in which the costs of living in the city equal those of not living in the city.
Workers and firms move to cities so they can be closer to each other, as both goods producers and input suppliers benefit from proximity.
In addition to lesser transport costs, workers have an easier time matching with firms and finding new jobs if they are located in the city.
The downsides of city living are congestion, higher land prices, and heightened competition.

While cities have always been vulnerable to attack, the nature of terrorism has changed in the last century or two.
What used to consist of assassinations of individual targets has turned into an indiscriminate spectacle of murder.
Now terrorism is as much, if not more, about the perceived risk or ``climate'' of terror as it is about raw casualties.

A natural policy response to a terrorist incident may be to try to decentralize, to get far away from dense urban areas where the probability of an attack is higher.
Indeed, Sheppard (2005) finds that a doubling of terrorist incidents can result in up to a 10\% reduction in urban development.

The structure of this paper is as follows.
First we review the general theory of urban structure and urban expansion.
Later we describe three seminal studies on the impact of terrorism on urban structure, offered by Rossi-Hansberg (2004), Harrigan and Martin (2002) and Sheppard (2005).
Finally, borrowing from the aforementioned analyses, we hypothesize that a terrorist attack may stoke the ``climate of terror," increase volatility, and slow down the urbanization rate.
Moreover, the type or magnitude of an attack may have varying effects on urbanization rates. An explosion may cause more fear than other weapons. Just as casualties affect society's apprehension, so does an attack's spectacle. This paper borrows much from Sheppard, but it is unique in that there has been little to no exploration into terrorism's effect on urbanization.


\section{\sc Theory}

\subsection{\sc General Model}
The determinants of urban structure and expansion include: the environment, demand for land, and policy.
The ideal model would take all three variables into account, but there have been few studies on the effect of climate, biomes or topography on urban structure.
Moreover, most studies that control for policy serve more as retroactive exercises in understanding, rather than as forecasts.

That said, we begin with an overview of the standard urban spatial equilibrium model.
Consider a city with $L$ households each with income $y$ and preferences stemming from a quasi-concave utility function $v(c,q)$, where $c$ denotes the consumption of a composite good and $q$ denotes the square units of housing.
Each household worker commutes $x$ units to the central business district (CBD).
At $x$ units of distance from the CBD, the worker's annual transportation costs are $t \times x$, and the cost per square unit of housing is $p(x)$.
In equilibrium, all households have equal utility 
$$
u = \max_{q} v(y - t \cdot x - q \cdot p(x), q)
$$
  
The housing production function $H(N, l)$ depends on capital $N$ and land $l$.
The equilibrium land rent function is $r(x)$ and the capital-land ratio (building density) is $S(x)$, such that 
$$\frac{\partial r(x)}{\partial x} < 0$$ 
and 
$$\frac{\partial S(x)}{\partial x} < 0$$
which means that land rents and building density decrease with distance from the CBD.
Derived from this is the population density $D(x,t,y,u)$, with the last three functional parameters being exogenously fed into the model.

The city's radius $\overline{x}$ is determined by housing developers' ability to bid land away from its alternative (i.e., agricultural) uses, denoted by $r_A$.
Thus, the city's maximum extent is given by $r(\overline{x}) = r_A$.

The model requires that all $L$ households fit within the urban periphery.
This is ensured by 
$$L = \int_{0}^{\overline{x}} \! 2 \pi \cdot \theta \cdot x \cdot D(x,t,y,u) \, \mathrm{d}x$$
where $\theta$ is the share of land available for public development.

In addition, firms in the CBD take capital and land as inputs to their production function $f(N,l)$.

As Sheppard so succinctly shows, we are left with the following hypotheses, enumerated in Table~\ref{generalHypotheses}.
\begin{table}[h!]
    \begin{tabular}{ | l | c | l | }\hline
    No. & Comparative Static & Parameter description                                                                    \\ \hline
    1   & $\partial \overline{x} / \partial L$                         & population                                                 \\ \hline
    2   & $\partial \overline{x} / \partial y$                         & household income                                           \\ \hline
    3   & $\partial \overline{x} / \partial t$                         & transport costs                                              \\ \hline
    4   & $\partial \overline{x} / \partial r_A$                         & price of undeveloped (agricultural) land                                     \\ \hline
    5   & $\partial \overline{x} / \partial H_l$                         & productivity of land in housing production     \\ \hline
    6   & $\partial \overline{x} / \partial \theta$                         & share of public land available for development                        \\ \hline
    7   & $\partial \overline{x} / \partial f_l$                         & productivity of land in production of export good \\ \hline
    8   & $\partial \overline{x} / \partial w$                         & global demand for export good                                \\ \hline
    \end{tabular}
    \caption{\label{generalHypotheses}Hypotheses of the standard urban equilibrium model.}
\end{table}


\subsection{\sc Rossi-Hansberg}
Rossi-Hansberg (2004) considers urban structure as well as the response to terrorism.
By studying firm interactions, he shows that the indefinite threat of a terrorist attack not only changes urban structure, but reduces output, wages, city population, residential capital investment, and the capital density gradient (i.e., the range of installed capital levels across locations within the city).
Moreover, he shows that the cost of a realized terrorist attack is more than just the investment necessary to rebuild destroyed structures; the loss of capital will incur a cost on agglomeration forces, and it will be more difficult to rebuild with a shrunken population and capital stock.

The threat of terrorism is modeled as a tax on transport and commuting costs, as agents experience more security checks, more congestion, and greater risk.
Since urban models are sensitive to changes in transport costs, terrorism can lead to substantial decreases in city density, resulting in a smaller city with more, but less compact, business districts. 

Because terrorists target locations with a high density of investment, firms are less likely to cluster.
This minimizes all of the productivity benefits that are concomitant with agglomeration, such as labor pooling and knowledge spillovers.
Because firms lack incentive to cluster together, they end up not clustering ``enough'' due to ``absentee landlords'' that cannot rent their land. 
One possible policy response to correct this inefficiency would be to implement a location specific capital subsidy, since terrorism acts like a location specific tax on capital investments.

Despite the short-run costs of a terrorist attack, Rossi-Hansberg's dynamic model leads to an optimistic conclusion: given enough time, wages, population and capital stocks will bounce back to their pre-attack levels.


\subsection{\sc Harrigan and Martin}
Harrigan and Martin (2002) use a labor pooling and core-periphery model to study the impact of terrorist incidents on urban centers.
They base their labor pooling model off of Krugman (1991) and their core-periphery model off of Fujita, Krugman, and Venables (1999).

\subsubsection{Labor Pooling Model}
Their labor pooling model considers firms, workers, and locations.
Firms locate where profits are highest, workers where wages are highest.
However, since firms determine wages, any uncertainty firms face leads to lower wages.
That is, fluctuations in demand or productivity leads to fluctuations in wages.
When firms have high productivity or face high demand, wages will be bid up; the opposite circumstances will cause wages to be bid down.
Wage variability will be lower in locations with more firms. 

In the labor pooling model, Harrigan and Martin find that a one-time terrorist attack has no long-run effect on city size.
This is because damage to infrastructure is only temporary.
They cite Davis and Weinstein, who studied the effects of the U.S. bombing of Hiroshima and showed that the relative size of cities in Japan was ultimately restored.\footnote{However, as Sheppard notes, war ends with a truce whereas the threat of terrorism is indefinite. For this reason, the impact of terrorism on urban structure cannot be compared with the impact of war on urban structure. The conclusion provided by Davis and Weinstein may have only been possible because the war stopped. Under the perpetual threat of terrorism, an urban system might not do as well.}

However, an \emph{ongoing} threat of terror incurs a cost borne by firms, such as higher insurance premiums, spending on security, security-induced delays, etc.
This acts as a kind of \emph{terror tax} -- adding to firms' cost of doing business in the city -- and causes some firms to leave the city, which reduces city labor demand and thus city wages.
Because city wages are now lower, workers flee the city, which causes wages and profits to equalize and results in a smaller city.
As a result, profits are lower in the city and in nonurban areas (due to land rents being bid up). 

As long as the terror tax does not rise above some threshold, there is still economic rationale for cities; that is, even in the wake of an attack, firms are still willing to pay for the benefits of agglomeration, as evidenced by city land rents that are often an order of magnitude higher than nonurban land rents.
From the worker's perspective, fear of terrorism reduces utility and causes workers to demand a wage premium.
The workers who continue to work in the city only do so because some firms are willing to pay them higher wages.

\subsubsection{Core-Periphery Model}
Unlike the labor pooling model, the core-periphery model considers transportation costs: it weighs the fact that firms prefer proximity to their customers and workers prefer proximity to their jobs. 

To see this, we consider two regions, A and B, which consists of two sectors: agriculture and manufacturing. 
Each region has an equal number of immobile farmers and an endogenously determined number of mobile, wage-maximizing manufacturers.

Since manufacturing firms -- which produce a variety of goods and experience increasing returns -- are subject to high transport costs, firms seek \emph{market access}.
This means firms have incentives to cluster in whichever region has the larger market. 
Similarly, workers have a preference for living in the city due to lower commuting costs and thus a lower cost of living.
However, the tradeoff becomes apparent as the city becomes more crowded, which drives up competition and reduces profits.

In a practical core-periphery model, there are two stable equilibria: one symmetric scenario in which the manufacturers are evenly split between the two regions, and the other in which all the workers end up in one region or the other.
Just as in the labor pooling model, the effect of terrorism is modeled as a terror tax borne by firms.
If, after a continuing terrorist threat, wages are still higher in the city, then there is no migration and firms continue to pay the agglomeration rent by offering workers higher wages.
On the other hand, if wages are lower in the city, workers and firms leave the city, until workers are split evenly between the two regions.
In this state, there are no agglomeration benefits, but neither is the risk of terrorism any higher.

In short, Harrigan and Martin find that a one-time terrorist incident has no long-run impact on the allocation of production and capital, but a persistent presence of terrorism can cause a long-run decrease in city size.
It is important to note their caveat about overestimating the effects of terrorism: while insurance premiums and terror taxes may seem disproportionately higher in the threatened city, to truly model the burden of urban insurance would require measuring the differential increase in costs between the affected city and the rest of the country.

\subsection{Sheppard}
For urban data, Sheppard (2005) uses a sample of 120 cities to represent the global urban population of cities with more than 100,000 persons.
For each city in the sample, he acquires two Landsat thematic mapper satellite images, one representing $T_1 \approx 1990$, the other representing $T_2 \approx 2000$.
The images allow him to classify each pixel as urban, water, or nonurban.
Furthermore, they allow him to distinguish between infill and ``outspill" development.
He interpolates missing population data with population growth rates to acquire a seamless set of data.
As an instrument, he includes each country's biome type.
For data on terrorist incidents, he uses the RAND database, which includes city-specific data for each incident.
However, to avoid endogeneity problems, he abstracts away from city-specific analysis and uses the number of terrorist incidents in a specific \emph{country} as his metric for terror.

While most of his methods are practical, some of his metrics and variables could be improved.
For instance, in order to model the hypothesis that high-density areas are more at risk, one regression discards any instances of terror that did not occur in a country's top five largest cities.
It would be interesting to tweak this metric slightly, and discard instances if they occurred in a city whose population was two standard deviations lower than the national primate.

Another example of metrics that could be improved are his use of certain variables to reflect model parameters.
For instance, to model the productivity of land in housing production, he uses the availability of a shallow groundwater aquifer; this seems sensible since it illustrates the cost of providing piped water to a new development.
To model the value of agricultural land, he uses national agricultural output per hectare; this is an acceptable metric.
However, to model global demand for a city's export good and a city's productivity in producing that export good, Sheppard uses the number of air linkages (the number of direct flights to out-of-country airports) and the national share of global IP address space.
While it may be unwise to analyze city-specific data on terrorist attacks -- since it would belittle the ``climate'' effect of terror -- here the analysis could benefit from better city-specific measures of global demand and especially productivity; the number of IP addresses allotted to a nation seems insufficient.

\section{\sc Data}
Unlike Sheppard, we use data on terrorist incidents collected from the Global Terrorism Database (GTD) instead of RAND.
The GTD offers more than twice as many instances and spans from 1970 to 2012.
In addition, it provides dozens of more variables describing a terrorist attack's success, weapons, perpetrators, targets, casualties, injuries, and more.

Furthermore, what sets this paper apart from all others is its analysis of the impact of terrorism on urbanization rates.
Whereas Sheppard analyzed the change in urban form -- that is, the difference in urban development between two periods -- we aim to analyze the rates at which rural workers migrate into the city.
Of course, this is the natural consequence of not being able to obtain satellite data, which would have created opportunities to study a few loose ends, such as the impact of terrorism on the capital density gradient. 

With urbanization rates in mind, we retrieve data from the World Bank Indicators dataset.
We look at 214 countries across 43 years, from 1970 to 2012.
The three urbanization statistics we consider are: absolute urban population, urbanization level (i.e., urban population as a percent of total country population), and annual urban population growth rate.

\subsection{Challenges}
There are several challenges we face when running a regression that draws from two different datasets.
First, there is discrepancy between country names: one dataset uses Ivory Coast, the other C\^{o}te d'Ivoir.
We fix all such discrepancies manually.

Second, some countries are present in one dataset but not in the other. 
The GTD makes a distinction between countries that have undergone name changes, such as New Hebrides (former name) and Vanuatu (present-day name).
If the World Bank has data on the present-day country name, we count terrorist incidents in the ``former'' country as incidents in the present-day country.

Third, the GTD makes a distinction between countries that have been dissolved and reunified, whereas the World Bank data does not.
For instance, the GTD documents terrorist attack data for West Germany, East Germany, and Germany, whereas the World Bank uses Germany.
Because this is a rare case of reunification -- the case of South Yemen and North Yemen reunifying into Yemen is another -- we treat terrorist incidents in West Germany and East Germany as having an impact on the urbanization of modern-day Germany.
In general, however, we ignore terrorist data for former countries that have undergone long, intricate histories of secession and dissolution.
For instance, we simply ignore attacks that occurred in Yugoslavia, instead of treating them as having occurred in Slovenia, Croatia, Bosnia and Herzegovina, Macedonia, Serbia, and Montenegro.
We do this for simplicity, and we acknowledge that it is an imperfection.

Fourth, we deliberate over how to treat terrorist attacks that occurred in overseas territories -- such as Corsica (France), the Falkland Islands (UK), and the Faroe Islands (Denmark).
We can either ignore these incidents, treat them as attacks in the ``mother'' country, or treat them as attacks in one or two of the most proximate nations.
We treat Corsica as part of France, as it has a large population and has a substantial terrorist presence.
By contrast, Guadeloupe or Martinique -- which both have populations larger than Corsica's -- are not treated as France, as they have a low threat of terrorism.
In general, our decisions are based on a territory's population size and threat level.

Fifth, we make a judgement call on city-states and countries with disputed autonomy.
Specifically, we count Vatican City as Italy and Taiwan as China.
This is not a reflection of our political opinions, but a consequence of lacking urbanization data on Vatican City and Taiwan.

Lastly, there is some misalignment over the way both datasets treat the United Kingdom.
The World Bank data tags its UK data with the Great Britain country code (GBR).
This would suggest that the World Bank really only has data on Great Britain: that is, Britain, Scotland, and Wales.
Because the GTD has an abundance of data on terrorist attacks occurring in Northern Ireland -- which was pulled from the Conflict Archive on the INternet -- we treat terrorist attacks in Northern Ireland as having an impact on the urbanization in Great Britain, which is what the World Bank treats as the United Kingdom.

\section{Analysis}


\section{Conclusion}


\begin{thebibliography}{9}

\bibitem{sheppard05}
  Stephen Sheppard,
  \emph{Urban Structure in a Climate of Terror}.
  Munich, Germany,
  2005.

\bibitem{harrigan02}
  Harrigan, J and Martin, P.
  \emph{Terrorism and the Resilience of Cities},
  Federal Reserve Board of New York Economic Policy Review,
  97-116,
  2002.

\bibitem{rossi04}
  Rossi-Hansberg, E.
  \emph{Cities under stress},
  Journal of Monetary Economics,
  51, 903-927,
  2004.

\end{thebibliography}


\end{document}
